{# This is a Jinja2 template for a latex file.  Both use braces heavily! 
    We use the raw ... endraw statements to surround untouched latex 
    and we have to escape literal braces.
#}
%<*driver>
%%!TEX TS-program = dtxmake
%%!TEX dtxmake-subengine = xelatexmk
\input docstrip.tex
\askforoverwritefalse
\generate{
    \file{\jobname.qns.tex}{\from{\jobname.dtx}{questions}}
    \file{\jobname.sol.tex}{\from{\jobname.dtx}{questions,solutions}}
}
\endbatchfile
%</driver>
%<*questions>
\documentclass{{ "{" }}article{{ "}" }}
\ProvidesFile
%<*dtx>
{{ "{" }}{{cookiecutter.exam_code}}.dtx{{ "}" }}
%</dtx>
%<questions&!solutions>{{ "{" }}{{cookiecutter.exam_code}}.qns.tex{{ "}" }}
%<questions&solutions>{{ "{" }}{{cookiecutter.exam_code}}.sol.tex{{ "}" }}
    [{% now 'local', '%Y/%m/%d' %} v0.0 {{cookiecutter.course_name}}, {{cookiecutter.term_name}}, {{cookiecutter.exam_name}}]
{{ [0,1,2,3,4,5,6,7,8,9]|random }}
\title{{ "{" }}{{cookiecutter.exam_name}}{{ "}" }}
\author{{ "{" }}{{cookiecutter.course_name}}{{ "}" }}
\date{{ "{" }}{{cookiecutter.exam_date | localize_date }}{{ "}" }}

\usepackage[insidebox
%<solutions>,answers
,nowatermark
]{automultiplechoice}
\usepackage{ifxetex}
\iftutex
  \usepackage{fontspec}
  \setmainfont{Frank Ruhl Libre}[
    ItalicFont=*,
    ItalicFeatures={FakeSlant}
  ]
  \setsansfont{Gotham Book}[
    BoldFont=Gotham Medium,
    ItalicFont=Gotham Book Italic
  ]          
\else
  \usepackage[T1]{fontenc}
  \usepackage[utf8]{inputenc}
  \usepackage{mathptmx}
  \usepackage{helvet}
\fi
\usepackage{titling}
\usepackage{changepage}%      For the `adjustwidth' environment

\usepackage{xcolor-nyu22}[2022/08/05]
% In the contemporary/subtle tone quadrant, we use sans for the main text and
% serif for the section titles.
%
% These font assignments have nothing to do with this package; they are part of
% the styling of a document
\usepackage{titlesec}
\newcommand{\headingfont}{\sffamily\color{NyuViolet}}
\titleformat*{\section}{\LARGE\headingfont}
\titleformat*{\subsection}{\Large\headingfont}
\titleformat*{\subsubsection}{\large\headingfont}
\renewcommand{\maketitlehooka}{\headingfont}

\usepackage{eso-pic}
\usepackage{graphicx}
\usepackage{etoolbox}
\newsavebox{\logobox}
\savebox{\logobox}{\includegraphics[]{nyu_short_black}}
\makeatletter
\renewcommand{\maketitle}{
    \noindent\parbox{\textwidth}{%
    \sffamily\color{UltraViolet}%
    \Large\@author
    \hfill\@title
    \hfill\@date}
    {\color{UltraViolet}\hrule\bigskip}
    \AddToShipoutPictureBG*{%
    \AtPageUpperLeft{%
      \raisebox{-1.5\ht\logobox}{\hspace{0.5\ht\logobox}\usebox{\logobox}}
      }%
    }
}
\makeatother
\usepackage{markversion}
% end document style



\usepackage{tikz}
\usetikzlibrary{MATH-UA-123-fall22}
\AtBeginDocument{\AMCboxDimensions{shape=oval,down=1mm}}

% Update the seed below to keep from copy & pasting the same one
% from quiz to quiz.

% This one was generated {% now 'local', '%c' %} 
\def\randomseed{{ "{" }}{{ random_int () }}{{ "}" }}

\AMCrandomseed{\randomseed} 
\pgfmathsetseed{\randomseed}
\usepackage{leincalc-fall22}
\usepackage{nicefrac}
\usepackage{cancel}
\usepackage{enumitem}
\usepackage{multicol}
\everymath{\displaystyle}

% \scoringDefaultS should be declared in the preamble.
\def\MCptval{2}
\def\TFptval{1}
\scoringDefaultS{e=0,v=0,b=(N==2 ? \TFptval : \MCptval),m=0}
\scoringDefaultM{b=\MCptval/N}
\AMCnumericOpts{scoreexact=1,scoreapprox=0}


% problem groups go here
% Remember!
% Question ID's cannot have dots in them, because then they become parts of multivalued fields.
% Also, they are typeset in text mode, so they can't have TeX special characters like underscores, either.
% Best: use a colon or hyphen to separate.
\element{TF}{
    \begin{question}{TF-cellphone}
    Cell phones may be used in class if they are kept on silent.
    \begin{choiceshoriz}[o]
%        \correctchoice[T]{True}\wrongchoice[F]{False}
        \wrongchoice[T]{True}\correctchoice[F]{False}
    \end{choiceshoriz}
\end{question}
}

\element{TF}{
    \begin{question}{TF-Piazza}
        The best place to ask (math) questions is the class's Piazza site.
        \begin{choiceshoriz}[o]
            \correctchoice[T]{True}
            \wrongchoice[F]{False}
        \end{choiceshoriz}
    \end{question}
}

\element{TF}{
    \begin{question}{TF-final}
        The final is cumulative.
        \begin{choiceshoriz}[o]
            \correctchoice[T]{True}
            \wrongchoice[F]{False}
        \end{choiceshoriz}
    \end{question}
}

    
\AMCboxColor{LightViolet1}
\def\AMCbeginQuestion#1#2{\par\noindent{\bf Question #1} #2\hspace*{1em}}



\begin{document}

\newenvironment{instructions}{\noindent\itshape}{}

\onecopy{1}{
\maketitle
\bgroup
% HEADER for name / section / ID
\setlength{\parindent}{0pt}%
%
This paper will be scanned and read by optical mark reading software.
Indicate your selections by filling in the circles \textbf{completely}.
If you wish to change your selection, erase your marks completely.
\bigskip

\begin{minipage}{0.4\textwidth}
    N Number:
    \par\smallskip
    \AMCcode{NNumber}{8}%
\end{minipage}
\hspace*{\fill}
\begin{minipage}{0.6\textwidth}\raggedright
    \namefield{\fbox{
        \begin{minipage}{0.95\textwidth}
            Name:
            \vspace*{.5cm}%\dotfill
            \vspace*{.5cm}%\dotfill
            \vspace*{1mm}
            \vspace{3ex} 
        \end{minipage}
    }}\medskip
    \noindent
    \fbox{
        \begin{minipage}{0.95\textwidth}
            NetID:
            \vspace*{.5cm}%\dotfill
            \vspace*{.5cm}%\dotfill
            \vspace*{1mm}
            \vspace{3ex} 
        \end{minipage}
    }
\end{minipage}
\hspace*{\fill}
\bigskip

This is a 20-minute quiz.  There are problems on the front and on the back.
\emph{Calculators are neither necessary nor allowed.}
% end of header group
\egroup
    
\bigskip


\begin{question}{MC-final}
    On which of these dates is the final exam?
    \begin{choices}
        \correctchoice{Tuesday, December 18}% correct
        \wrongchoice{Wednesday, December 19}% calculus exams
        \wrongchoice{Monday, December 17}% review session
        \wrongchoice{Thursday, December 20}% commencement
        \wrongchoice{Friday, December 14}% last day of class
    \end{choices}
\end{question}


\begin{questionmult}{MC-cheating}
    Which of these are examples of academic dishonesty?
    \begin{choices}
        \correctchoice{Copying the homework of another student without collaboration and citation}
        \correctchoice{Falsifying documents to justify makeup exams}
        \correctchoice{Placing an ad on craiglist seeking someone to take your exam}
        \correctchoice{Asking homework questions on Yahoo! Answers and using answers found there as your own work}
    \end{choices}
\end{questionmult}

\noindent\emph{Identify each of the following statements as true or false.}

\shufflegroup{TF}
\insertgroup{TF}

\clearpage

\begin{question}{FR-why}
    Why are you taking this course?  What would you like to learn?
    \AMCOpen{lines=6,dots=false,answer={%
        }
    }{
        \wrongchoice[0]{0}\scoring{0}
        \wrongchoice[1]{1}\scoring{1}
        \correctchoice[2]{2}\scoring{2}
    }    
\end{question}

\begin{question}{FR-map}
    Draw a map of your home town, and indicate on the map where you live.
    \AMCOpen{lines=7,dots=false,answer={%
        }
    }{
        \wrongchoice[0]{0}\scoring{0}
        \wrongchoice[1]{1}\scoring{1}
        \wrongchoice[2]{2}\scoring{2}
        \correctchoice[3]{3}\scoring{3}
    }    
\end{question}

    
}% end of \onecopy group
\end{document}
   
%</questions>
